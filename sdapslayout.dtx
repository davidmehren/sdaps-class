% \iffalse meta-comment
%
% Copyright (C) 2014 by Henry Menke <henrimenke@gmail.com>
% Copyright (C) 2015-2016 by Benjamin Berg <benjamin@sipsolutions.net>
%
% This work may be distributed and/or modified under the
% conditions of the LaTeX Project Public License, either version 1.3
% of this license or (at your option) any later version.
% The latest version of this license is in
%   http://www.latex-project.org/lppl.txt
% and version 1.3 or later is part of all distributions of LaTeX
% version 2005/12/01 or later.
%
% This work has the LPPL maintenance status `maintained'.
% 
% The Current Maintainer of this work is Benjamin Berg.
%
% \fi
%
% \iffalse
%<*driver>
\ProvidesFile{sdapslayout.dtx}
%</driver>
%<package>\NeedsTeXFormat{LaTeX2e}[1999/12/01]
%<package>\ProvidesPackage{sdapslayout}
%<*package>
    [2015/04/10 v0.1 Initial version of SDAPS layout package]
%</package>
%
%<*driver>
\documentclass{l3doc}
\usepackage{framed,lipsum}
\usepackage{sdapslayout}[2015/04/10]
\usepackage{xcolor}
\usepackage{listings}
\usepackage{enumitem}
\usepackage{todonotes}
\EnableCrossrefs
%\CodelineIndex
\RecordChanges
\begin{document}
  \DocInput{sdapslayout.dtx}
\end{document}
%</driver>
% \fi
%
% \CheckSum{0}
%
% \CharacterTable
%  {Upper-case    \A\B\C\D\E\F\G\H\I\J\K\L\M\N\O\P\Q\R\S\T\U\V\W\X\Y\Z
%   Lower-case    \a\b\c\d\e\f\g\h\i\j\k\l\m\n\o\p\q\r\s\t\u\v\w\x\y\z
%   Digits        \0\1\2\3\4\5\6\7\8\9
%   Exclamation   \!     Double quote  \"     Hash (number) \#
%   Dollar        \$     Percent       \%     Ampersand     \&
%   Acute accent  \'     Left paren    \(     Right paren   \)
%   Asterisk      \*     Plus          \+     Comma         \,
%   Minus         \-     Point         \.     Solidus       \/
%   Colon         \:     Semicolon     \;     Less than     \<
%   Equals        \=     Greater than  \>     Question mark \?
%   Commercial at \@     Left bracket  \[     Backslash     \\
%   Right bracket \]     Circumflex    \^     Underscore    \_
%   Grave accent  \`     Left brace    \{     Vertical bar  \|
%   Right brace   \}     Tilde         \~}
%
%
% \changes{v0.1}{2015/01/14}{Initial version}
%
% \GetFileInfo{sdapslayout.dtx}
%
% \DoNotIndex{\newcommand,\newenvironment}
% 
%
% \title{The \textsf{sdapslayout} package\thanks{This document
%   corresponds to \textsf{sdapslayout}~\fileversion, dated \filedate.}}
% \author{Benjamin Berg \\ \texttt{benjamin@sipsolutions.net}}
%
% \maketitle
%
% \section{Introduction}
%
% \begin{documentation}
%
% This package provides a set of more complex layouting options on top of the
% \texttt{sdapsarray} package. The following environments are provided:
% \begin{itemize}
%   \item \env{choicearray}
%   \item \env{rangearray}
% \end{itemize}
%
% The \env{rangearray} is not quite as powerful, as it does not expose some
% of the \env{sdapsarray} options to the user. Note that much more complex
% custom layouts can be created directly with \env{sdapsarray}.
%
% \subsection{\env{choicearray} environment}
%
% \DescribeEnv{choicearray}
% The \env{choicearray} environment represents a tabular layout for a set of multiple
% choice questions which have the same possible answers. A new header is
% created in the metadata to group the questions. The header is currently
% \emph{not shown} in the PDF and it cannot contain fragile content due to
% implementation constraints!
%
%   \lstset{language=,gobble=1,basicstyle=\small,breaklines=true,numbers=left,frame=lr}
%
% \begin{figure}
%   \begin{minipage}{0.65\linewidth}
%   \begin{lstlisting}
%     \begin{choicearray}[layouter=rotated]{A group of questions}
%       \choice{Choice 1}
%       \choice{Choice 2}
%       \question{Question one}
%       \question{Question two}
%     \end{choicearray}
%   \end{lstlisting}
%   \end{minipage}
%   \begin{minipage}{0.34\linewidth}
%     \begin{choicearray}[layouter=rotated]{A group of questions}
%       \choice{Choice 1}
%       \choice{Choice 2}
%       \question{Question one}
%       \question{Question two}
%     \end{choicearray}
%   \end{minipage}
%   \label{choicearray-example}
%   \caption{Example of \env{choicearray} environment}
% \end{figure}
%
% \begin{figure}
%   \begin{minipage}{0.65\linewidth}
%   \begin{lstlisting}
%     \begin{choicearray}[vertical,layouter=rotated]{A group of questions}
%       \choice{Choice 1}
%       \choice{Choice 2}
%       \question{Question one}
%       \question{Question two}
%     \end{choicearray}
%   \end{lstlisting}
%   \end{minipage}
%   \begin{minipage}{0.34\linewidth}
%     \begin{choicearray}[vertical,layouter=rotated]{A group of questions}
%       \choice{Choice 1}
%       \choice{Choice 2}
%       \question{Question one}
%       \question{Question two}
%     \end{choicearray}
%   \end{minipage}
%   \label{choicearray-example}
%   \caption{Example of vertical \env{choicearray} environment}
% \end{figure}
%
% The following optional \oarg{keyword parameters} can be passed to the
% environment. These keyword parameters default to the values provided in the
% SDAPS environment and can for example be modified globally for the document.
%
% \todo[inline]{Right now all arguments can be overriden, this should not be
%    the case. It would be correct if the variable name is loaded from the
%    kwargs, and then the defaults for align and horizontal/vertical should be
%    applied}
%
% \begin{description}[style=nextline]
%   \item[align] named alignment group, by default \env{choicearay} environments are aligned if the layouter and orientation match
%   \item[horizontal] set horizontal mode where each question is one row (default)
%   \item[layouter] set the \texttt{sdapsarray} layouter (e.g. \texttt{rotated} for rotated column headers)
%   \item[text] override the normal header (not useful currently!)
%   \item[var] if set appends the variable name to the newly created scope (`_' separator is added automatically)
%   \item[vertical] set vertical mode where each question is one column
% \end{description}
%
% Inside the environment you need to first define all possible answers with
% \cs{choice} macro and each question using the \cs{question} macro.
%
% \subsubsection{\cs{choice} command}
%
% \DescribeMacro{\choice}
% The \cs{choice} macro is of the form \cs{choice}\oarg{options}\marg{text}.
% The given text is used as the label for the choice. This text may contain
% any \LaTeX{} macro (including fragile commands) if the \texttt{text} optional
% keyword argument was given. Additionally you can set a variable name for this
% option using the \texttt{var} keyword argument. If no keyword argument is
% given then the answers will be auto enumerated.
%
% \begin{description}[style=nextline]
%   \item[text] override the text which is exported, if set the \marg{text} argument may contain any content \emph{including fragile content}
%   \item[var] if set uses the variable name for the checkbox instead of its numeric ID
% \end{description}
%
% \subsubsection{\cs{question} command}
%
% \DescribeMacro{\question}
% The \cs{question} macro is of the form \cs{question}\oarg{options}\marg{text}.
% The given text is used as the label for the question. This text may contain
% any \LaTeX{} macro (including fragile commands) if the \texttt{text} optional
% keyword argument was given. Additionally you can set a variable name for this
% option using the \texttt{var} keyword argument. If no keyword argument is
% given then the questions will be auto enumerated.
%
% \begin{description}[style=nextline]
%   \item[text] override the text which is exported, if set the \marg{text} argument may contain any content \emph{including fragile content}
%   \item[var] if set uses the variable name for the checkbox instead of its numeric ID
% \end{description}
%
%
% \subsection{\env{rangearray} environment}
%
% \DescribeEnv{rangearray}
% The \env{rangearray} environment represents a tabular layout for a set of
% single choice answers. The questions in this environment are always layed out
% as one question per line. For each question the user pick an item on a range
% or an optional alternative answer.
%
% \begin{figure}
%   \begin{framed}
%     \begin{rangearray}[count=7,other]{A group of questions}
%       \range{Question one}{from}{to}{other a}
%       \range{Question two}{lower}{upper}{other b}
%     \end{rangearray}
%   \end{framed}
%
%   \begin{lstlisting}
%     \begin{rangearray}[count=7,other]{A group of questions}
%       \range{Question one}{from}{to}{other a}
%       \range{Question two}{lower}{upper}{other b}
%     \end{rangearray}
%   \end{lstlisting}
%
%   \label{choicearray-example}
%   \caption{Example of \env{rangearray} environment}
% \end{figure}
%
% The following optional \oarg{keyword parameters} can be passed to the
% environment. These keyword parameters default to the values provided in the
% SDAPS environment and can for example be modified globally for the document.
%
% \todo[inline]{Right now all arguments can be overriden, this should not be
%    the case. It would be correct if the variable name is loaded from the
%    kwargs, and then the defaults for align and count should be applied}
%
% \begin{description}[style=nextline]
%   \item[align] named alignment group, by default \env{rangearray} environments are aligned if the \cs{other} flag is identical
%   \item[count] the number of choices (default: 5)
%   \item[other] whether the user has a further option outside of the range (default: false)
%   \item[text] override the normal header (not useful currently!)
%   \item[var] if set appends the variable name to the newly created scope (`_' separator is added automatically)
% \end{description}
%
%
%
% \section{Usage}
%
% Put text here.
%
% \end{documentation}
%
% \section{Implementation}
%
% This package uses the \LaTeX3 language internally, so we need to enable it.
%    \begin{macrocode}
% We need at least 2011-08-23 for \keys_set_known:nnN
\RequirePackage{expl3}[2011/08/23]
%\RequirePackage{xparse}
\ExplSyntaxOn
%    \end{macrocode}
%
% And we need a number of other packages.
%    \begin{macrocode}
\ExplSyntaxOff

\RequirePackage{sdapsbase}
\RequirePackage{sdapsarray}
\RequirePackage{xparse}


\ExplSyntaxOn

%    \end{macrocode}
%
%%%%%%%%%%%%%%%%%%%%%%%%%%%%%%%%%%%%%%%%%%%%%%%%%%%%%%%
%
%
% \subsection{Choice Question Layout}
%
% \subsubsection{Choice Question Matrix Layout}
%
% The following macros provide the funcitonality to layout choice questions in
% a matrix like fashion.
%
%    \begin{macrocode}


\bool_new:N \l_sdaps_choicearray_horizontal_bool
\tl_new:N \l_sdaps_choicearray_var_tl
\tl_new:N \l_sdaps_choicearray_text_tl
\tl_new:N \l_sdaps_choicearray_layouter_tl
\tl_new:N \l_sdaps_choicearray_align_tl
\tl_new:N \l_sdaps_choice_var_tl
\tl_new:N \l_sdaps_choice_text_tl
\tl_new:N \l_sdaps_question_var_tl
\tl_new:N \l_sdaps_question_text_tl
\seq_new:N \g_sdaps_choices_var_seq
\seq_new:N \g_sdaps_choices_text_seq

\keys_define:nn { sdaps / choicearray }
{
  horizontal .bool_set:N = \l_sdaps_choicearray_horizontal_bool,
  horizontal .default:n  = true,
  horizontal .initial:n  = true,
  vertical   .bool_set_inverse:N = \l_sdaps_choicearray_horizontal_bool,
  vertical   .default:n  = true,
  var        .tl_set:N   = \l_sdaps_choicearray_var_tl,
  text       .tl_set:N   = \l_sdaps_choicearray_text_tl,
  layouter   .tl_set:N   = \l_sdaps_choicearray_layouter_tl,
  layouter   .initial:n  = default,
  align      .tl_set:N   = \l_sdaps_choicearray_align_tl,
  align      .initial:n  = { choicearray\bool_if:NTF\l_sdaps_choicearray_horizontal_bool{horizontal}{vertical}\tl_use:N\l_sdaps_choicearray_layouter_tl },
}

\keys_define:nn { sdaps / choicearray / choice }
{
  var        .tl_set:N   = \l_sdaps_choice_var_tl,
  text       .tl_set:N   = \l_sdaps_choice_text_tl,
}

\keys_define:nn { sdaps / choicearray / question }
{
  var        .tl_set:N   = \l_sdaps_question_var_tl,
  text       .tl_set:N   = \l_sdaps_question_text_tl,
}

\cs_new_protected_nopar:Npn \_sdaps_choicearray_preprocess:nn #1#2
{
  \keys_set:nn { sdaps / choicearray } { #1 }

  \tl_if_empty:NTF \l_sdaps_choicearray_text_tl {
    \sdaps_qobject_begin:nnn { choicearray } { Head } { #2 }
  } {
    \sdaps_qobject_begin:nnV { choicearray } { Head } \l_sdaps_choicearray_text_tl
  }

  \tl_if_empty:NF \l_sdaps_choicearray_var_tl {
    \sdaps_context_append:nVn { var } \l_sdaps_choicearray_var_tl { _ }
  }
}
\cs_generate_variant:Nn \_sdaps_choicearray_preprocess:nn { Vn }

\cs_new_protected_nopar:Npn \_sdaps_choicearray_postprocess:
{
  \sdaps_qobject_end:n { choicearray }
}

\cs_new_protected_nopar:Npn \_sdaps_choicearray_process_choice_insert_tail_after:w {
  \bgroup
    \group_insert_after:N \_sdaps_choicearray_process_choice_tail:
    \sdaps_array_nested_alignenv:
    \tex_let:D\next=
}

\cs_new_protected_nopar:Nn \_sdaps_choicearray_process_choice_tail: {
  \ignorespaces
}

\cs_new_nopar:Nn \_sdaps_choicearray_grab_choice:n {
  \seq_gput_right:Nn \g_sdaps_choices_text_seq { #1 }
  \group_begin:
    \sdaps_array_nested_alignenv:
    #1
  \group_end:
  \_sdaps_choicearray_process_choice_tail:
}

\cs_new_nopar:Npn \_sdaps_choicearray_process_choice:nw #1
{
  % This modifies grouping so it has to be at the start
  \sdaps_array_alignment:

  \tl_clear:N \l_sdaps_choice_var_tl
  \tl_clear:N \l_sdaps_choice_text_tl

  \keys_set:nn { sdaps / choicearray / choice } { #1 }

  \seq_gput_right:Nx \g_sdaps_choices_var_seq { \l_sdaps_choice_var_tl }

  \tl_if_empty:NTF \l_sdaps_choice_text_tl {
    % We need to leave a command in the stream that grabs the next parameter
    % and outputs it immediately
    \cs_set_eq:NN \l_tmpa_token \_sdaps_choicearray_grab_choice:n
  } {
    % Nothing else to do
    \seq_gput_right:NV \g_sdaps_choices_text_seq { \l_sdaps_choice_text_tl }
    \cs_set_eq:NN \l_tmpa_token \_sdaps_choicearray_process_choice_insert_tail_after:w
  }
  \l_tmpa_token
}
\cs_generate_variant:Nn \_sdaps_choicearray_process_choice:nw { Vw }

\cs_new_protected_nopar:Nn \_sdaps_choicearray_process_question_grab:n {
  \tl_set:Nn \l_sdaps_question_text_tl { #1 }

  \_sdaps_choicearray_process_question_head:

  \group_begin:
    \sdaps_array_nested_alignenv:
    #1
  \group_end:

  \_sdaps_choicearray_process_question_tail:
}

\cs_new_protected_nopar:Npn \_sdaps_choicearray_process_question_inline:w {
  \_sdaps_choicearray_process_question_head:
  \bgroup
    \group_insert_after:N \_sdaps_choicearray_process_question_tail:
    \sdaps_array_nested_alignenv:
    \tex_let:D\next=
}

\cs_new_protected_nopar:Nn \_sdaps_choicearray_process_question_head: {
  \sdaps_qobject_begin:nnV { choicearray_question } { Choice } \l_sdaps_question_text_tl

  \tl_if_empty:NF \l_sdaps_question_var_tl {
   \sdaps_context_append:nVn { var } \l_sdaps_question_var_tl { _ }
  }
}

\cs_new_protected_nopar:Nn \_sdaps_choicearray_process_question_tail: {
  \seq_map_inline:Nn \g_sdaps_choices_text_seq {
    \sdaps_answer:f { ##1 }
  }

  \seq_map_inline:Nn \g_sdaps_choices_var_seq {
    \sdaps_array_alignment:
    \sdaps_checkbox:nn { ##1 } {  }
  }

  \sdaps_qobject_end:n { choicearray_question }
  \ignorespaces
}

\cs_new_nopar:Npn \_sdaps_choicearray_process_question:nw #1
{
  \sdaps_array_newline:

  \keys_set:nn { sdaps / choicearray / question } { #1 }

  \tl_if_empty:NTF \l_sdaps_question_text_tl {
    % We need to leave a command in the stream that grabs the next parameter,
    % outputs it again, and finishes the question.
    \cs_set_eq:NN \l_tmpa_token \_sdaps_choicearray_process_question_grab:n
  } {
    % Insert the question around the next argument
    \cs_set_eq:NN \l_tmpa_token \_sdaps_choicearray_process_question_inline:w
  }
  \l_tmpa_token
}
\cs_generate_variant:Nn \_sdaps_choicearray_process_question:nw { Vw }



%
%    \end{macrocode}
%
%
% \subsection{Range Question Layout}
%
% \subsubsection{Range Question Matrix Layout}
%
% The following macros provide the functionality to layout range/option
% questions in a matrix like fashion.
%
%    \begin{macrocode}


\tl_new:N \l_sdaps_rangearray_var_tl
\tl_new:N \l_sdaps_rangearray_text_tl
\tl_new:N \l_sdaps_rangearray_align_tl
\int_new:N \l_sdaps_rangearray_rangecount_int
\bool_new:N \l_sdaps_rangearray_other_bool
\tl_new:N \g_sdaps_question_var_tl
\tl_new:N \g_sdaps_question_text_tl
\tl_new:N \g_sdaps_question_lowertext_tl
\tl_new:N \g_sdaps_question_uppertext_tl
\tl_new:N \g_sdaps_question_othertext_tl

\keys_define:nn { sdaps / rangearray }
{
  var        .tl_set:N   = \l_sdaps_rangearray_var_tl,
  text       .tl_set:N   = \l_sdaps_rangearray_text_tl,
  count      .int_set:N  = \l_sdaps_rangearray_rangecount_int,
  count      .initial:n  = 5,
  align      .tl_set:N   = \l_sdaps_rangearray_align_tl,
  align      .initial:n  = { rangearray\bool_if:NTF\l_sdaps_rangearray_other_bool{optcol}{nooptcol}\int_use:N\l_sdaps_rangearray_rangecount_int },
  other      .bool_set:N = \l_sdaps_rangearray_other_bool,
  other      .default:n  = true,
  other      .initial:n  = false,
}

\keys_define:nn { sdaps / rangearray / question }
{
  var        .tl_gset:N   = \g_sdaps_question_var_tl,
  text       .tl_gset:N   = \g_sdaps_question_text_tl,
  upper      .tl_gset:N   = \g_sdaps_question_uppertext_tl,
  lower      .tl_gset:N   = \g_sdaps_question_lowertext_tl,
  other      .tl_gset:N   = \g_sdaps_question_othertext_tl,
}

\cs_new_protected_nopar:Npn \_sdaps_rangearray_preprocess:nn #1#2
{
  \keys_set:nn { sdaps / rangearray } { #1 }

  \tl_if_empty:NTF \l_sdaps_rangearray_text_tl {
    \sdaps_qobject_begin:nnn { rangearray } { Head } { #2 }
  } {
    \sdaps_qobject_begin:nnV { rangearray } { Head } \l_sdaps_rangearray_text_tl
  }

  \sdaps_context_set:n {checkbox={ellipse}}

  \tl_if_empty:NF \l_sdaps_rangearray_var_tl {
    \sdaps_context_append:nVn { var } \l_sdaps_rangearray_var_tl { _ }
  }
}
\cs_generate_variant:Nn \_sdaps_rangearray_preprocess:nn { Vn }

\cs_new_protected_nopar:Npn \_sdaps_rangearray_postprocess:
{
  \sdaps_qobject_end:n { rangearray }
}

% Before/After the different parts

\cs_new_protected_nopar:Nn \_sdaps_rangearray_process_question_before_question: {
  \sdaps_array_newline:
  % Note: This needs to be after sdaps_array_newline as the command may be
  %       discarded otherwise (i.e. it does not make it into the output stream)
  \sdaps_qobject_begin:nnV { rangearray_question } { Range } \g_sdaps_question_text_tl

  \tl_if_empty:NF \g_sdaps_question_var_tl {
   \sdaps_context_append:nVn { var } \g_sdaps_question_var_tl { _ }
  }

  \ignorespaces
}

\cs_new_protected_nopar:Nn \_sdaps_rangearray_process_question_before_lower: {
  % right align
  \sdaps_array_alignment:
  \sdaps_range:nnV { lower } { 0 } \g_sdaps_question_lowertext_tl
  \hfill
  \ignorespaces
}

\cs_new_protected_nopar:Nn \_sdaps_rangearray_process_question_before_upper: {
  \sdaps_array_alignment:
  \sdaps_range:nnV { upper } { \l_sdaps_rangearray_rangecount_int - 1 } \g_sdaps_question_uppertext_tl
  \ignorespaces
}

\cs_new_protected_nopar:Nn \_sdaps_rangearray_process_question_before_other: {
  \sdaps_array_alignment:
  \sdaps_answer:V \g_sdaps_question_othertext_tl
  \sdaps_checkbox:nn { dummy } { 0 } {} ~ {}
  \ignorespaces
}



\cs_new_protected_nopar:Nn \_sdaps_rangearray_process_question_after_question: {
  \tl_if_empty:NTF \g_sdaps_question_lowertext_tl {
    \cs_set_eq:NN \l_tmpa_token \_sdaps_rangearray_process_question_grab_lower:n
  } {
    \cs_set_eq:NN \l_tmpa_token \_sdaps_rangearray_process_question_inline_lower:w
  }
  \l_tmpa_token
}

\cs_new_protected_nopar:Nn \_sdaps_rangearray_process_question_after_lower: {
  % Insert the option checkbox column
  \sdaps_array_alignment:
  % Assume we have at least one checkbox
  \sdaps_checkbox:nn { dummy } { 1 }
  \int_step_inline:nnnn { 2 } { 1 } { \l_sdaps_rangearray_rangecount_int } {
    \hspace{1em} \sdaps_checkbox:nn { dummy } { ##1 }
  }

  \tl_if_empty:NTF \g_sdaps_question_uppertext_tl {
    \cs_set_eq:NN \l_tmpa_token \_sdaps_rangearray_process_question_grab_upper:n
  } {
    \cs_set_eq:NN \l_tmpa_token \_sdaps_rangearray_process_question_inline_upper:w
  }
  \l_tmpa_token
}

\cs_new_protected_nopar:Nn \_sdaps_rangearray_process_question_after_upper: {
  \hfill\kern 0pt

  \bool_if:NTF \l_sdaps_rangearray_other_bool {
    \tl_if_empty:NTF \g_sdaps_question_othertext_tl {
      \cs_set_eq:NN \l_tmpa_token \_sdaps_rangearray_process_question_grab_other:n
    } {
      \cs_set_eq:NN \l_tmpa_token \_sdaps_rangearray_process_question_inline_other:w
    }
  } {
    \cs_set_eq:NN \l_tmpa_token \_sdaps_rangearray_process_question_finish:
  }
  \l_tmpa_token
}

\cs_new_protected_nopar:Nn \_sdaps_rangearray_process_question_after_other: {
  \hfill\kern 0pt
  \ignorespaces
}

\cs_new_protected_nopar:Nn \_sdaps_rangearray_process_question_finish: {
  \sdaps_qobject_end:n { rangearray_question }
  \ignorespaces
}

% Processors for inline processing/grabbing the argument

\cs_new_protected_nopar:Nn \_sdaps_rangearray_process_question_grab_question:n {
  \tl_gset:Nn \g_sdaps_question_text_tl { #1 }

  \_sdaps_rangearray_process_question_before_question:
  \group_begin:
    \sdaps_array_nested_alignenv:
    #1
  \group_end:
  \_sdaps_rangearray_process_question_after_question:
}

\cs_new_protected_nopar:Nn \_sdaps_rangearray_process_question_grab_lower:n {
  \tl_gset:Nn \g_sdaps_question_lowertext_tl { #1 }

  \_sdaps_rangearray_process_question_before_lower:
  \group_begin:
    \sdaps_array_nested_alignenv:
    #1
  \group_end:
  \_sdaps_rangearray_process_question_after_lower:
}

\cs_new_protected_nopar:Nn \_sdaps_rangearray_process_question_grab_upper:n {
  \tl_gset:Nn \g_sdaps_question_uppertext_tl { #1 }

  \_sdaps_rangearray_process_question_before_upper:
  \group_begin:
    \sdaps_array_nested_alignenv:
    #1
  \group_end:
  \_sdaps_rangearray_process_question_after_upper:
}

\cs_new_protected_nopar:Nn \_sdaps_rangearray_process_question_grab_other:n {
  \tl_gset:Nn \g_sdaps_question_othertext_tl { #1 }

  % If the text is empty, assume that this particular question does not have
  % an alternative choice. Note that this column might not exist and this
  % macro will not even be called in that case.
  % If we skip the optional "other" option then we still need to insert the
  % alignment to create the column.
  \tl_if_empty:NTF \g_sdaps_question_othertext_tl {
    \sdaps_array_alignment:
    \ignorespaces
  } {
    \_sdaps_rangearray_process_question_before_other:
    \group_begin:
      \sdaps_array_nested_alignenv:
      #1
    \group_end:
    \_sdaps_rangearray_process_question_after_other:
  }
  \_sdaps_rangearray_process_question_finish:
}



\cs_new_protected_nopar:Npn \_sdaps_rangearray_process_question_inline_question:w {
  \_sdaps_rangearray_process_question_before_question:
  \bgroup
    \group_insert_after:N \_sdaps_rangearray_process_question_after_question:
    \sdaps_array_nested_alignenv:
    \tex_let:D\next=
}

\cs_new_protected_nopar:Npn \_sdaps_rangearray_process_question_inline_lower:w {
  \_sdaps_rangearray_process_question_before_lower:
  \bgroup
    \group_insert_after:N \_sdaps_rangearray_process_question_after_lower:
    \tex_let:D\next=
}

\cs_new_protected_nopar:Npn \_sdaps_rangearray_process_question_inline_upper:w {
  \_sdaps_rangearray_process_question_before_upper:
  \bgroup
    \group_insert_after:N \_sdaps_rangearray_process_question_after_upper:
    \tex_let:D\next=
}

\cs_new_protected_nopar:Npn \_sdaps_rangearray_process_question_inline_other:w {
  \_sdaps_rangearray_process_question_before_other:
  % If we reach this macro then a text has been set for the other item. This
  % means we never need to ignore the "other" parameter at this point.
  \bgroup
    \group_insert_after:N \_sdaps_rangearray_process_question_after_other:
    \group_insert_after:N \_sdaps_rangearray_process_question_finish:
    \tex_let:D\next=
}




\cs_new_nopar:Npn \_sdaps_rangearray_process_question:nw #1
{
  % Is there a better way other than clearing these before parsing?
  \tl_gclear:N \g_sdaps_question_var_tl
  \tl_gclear:N \g_sdaps_question_text_tl
  \tl_gclear:N \g_sdaps_question_uppertext_tl
  \tl_gclear:N \g_sdaps_question_lowertext_tl
  \tl_gclear:N \g_sdaps_question_othertext_tl

  \keys_set:nn { sdaps / rangearray / question } { #1 }

  \tl_if_empty:NTF \g_sdaps_question_text_tl {
    % We need to leave a command in the stream that grabs the next parameter,
    % outputs it again, and finishes the question.
    \cs_set_eq:NN \l_tmpa_token \_sdaps_rangearray_process_question_grab_question:n
  } {
    % We need to generate the question after the next group stops
    \cs_set_eq:NN \l_tmpa_token \_sdaps_rangearray_process_question_inline_question:w
  }
  \l_tmpa_token
}
\cs_generate_variant:Nn \_sdaps_rangearray_process_question:nw { Vw }



%
%    \end{macrocode}
%
%
%
% \subsection{Export user facing environments}
%
%    \begin{macrocode}
%


\newenvironment { choicearray } [ 2 ] []
{
  \group_begin:

    \sdaps_context_get:nN { choicearray } \l_tmpa_tl
    \tl_if_eq:NNT \l_tmpa_tl \q_no_value {
      \tl_set:Nn \l_tmpa_tl {}
    }

    \tl_if_empty:nF { #1 } {
      \tl_if_empty:NTF \l_tmpa_tl {
        \tl_set:Nn \l_tmpa_tl { #1 }
      } {
        \tl_set:Nf \l_tmpa_tl { \l_tmpa_tl, #1 }
      }
    }

    \_sdaps_choicearray_preprocess:Vn \l_tmpa_tl { #2 }
    % Clear the variables
    \seq_gclear:N \g_sdaps_choices_var_seq
    \seq_gclear:N \g_sdaps_choices_text_seq

    % Define new commands
    \newcommand \choice [ 1 ] [] {
      \_sdaps_choicearray_process_choice:nw { ##1 }
    }
    \newcommand \question [ 1 ] [] {
      \_sdaps_choicearray_process_question:nw { ##1 }
    }

    \group_begin:

      \tl_set:Nx \l_tmpb_tl {keepenv,layouter=\tl_use:N\l_sdaps_choicearray_layouter_tl,align=\l_sdaps_choicearray_align_tl\bool_if:NF\l_sdaps_choicearray_horizontal_bool{,flip}}
      \expandafter\sdapsarray\expandafter[\l_tmpb_tl]
}
{
      \endsdapsarray
    \group_end:
    % Process keys
    \_sdaps_choicearray_postprocess:

  \group_end:
}



\newenvironment { rangearray } [ 2 ] []
{
  \group_begin:

    \sdaps_context_get:nN { rangearray } \l_tmpa_tl
    \tl_if_eq:NNT \l_tmpa_tl \q_no_value {
      \tl_set:Nn \l_tmpa_tl {}
    }

    \tl_if_empty:nF { #1 } {
      \tl_if_empty:NTF \l_tmpa_tl {
        \tl_set:Nn \l_tmpa_tl { #1 }
      } {
        \tl_set:Nf \l_tmpa_tl { \l_tmpa_tl, #1 }
      }
    }

    \_sdaps_rangearray_preprocess:Vn \l_tmpa_tl { #2 }

    \newcommand \range [ 1 ] [] {
      \_sdaps_rangearray_process_question:nw { ##1 }
    }

    \group_begin:

      \tl_set:Nx \l_tmpb_tl {keepenv,align=\l_sdaps_rangearray_align_tl}
      \expandafter\sdapsarray\expandafter[\l_tmpb_tl]
}
{
      \endsdapsarray
    \group_end:
    % Process keys
    \_sdaps_rangearray_postprocess:

  \group_end:
}


\ExplSyntaxOff

%
%    \end{macrocode}
%
% \iffalse
% \PrintChanges
% \PrintIndex
% \fi
%
% \Finale
\endinput
